\documentclass[12pt]{fphw}

\usepackage[utf8]{inputenc} % Required for inputting international characters
\usepackage[T1]{fontenc} % Output font encoding for international characters
\usepackage{mathpazo} % Use the Palatino font

\usepackage{graphicx} % Required for including images

\usepackage{booktabs} % Required for better horizontal rules in tables

\usepackage{listings} % Required for insertion of code

\usepackage{enumerate} % To modify the enumerate environment

\usepackage{amsmath}
\usepackage{amssymb}
\usepackage{amsthm}

\title{Logic Coursework 2024/25: Written Work}
\author{Huseyin Emre Ozden}
\date{March 25th, 2025}
\institute{Durham University}
\class{Computational Thinking}
\professor{Prof. Barnaby Martin}

\begin{document}

\maketitle

\section*{Question 1}

\begin{problem}
  Answer the following questions about complete sets of logical connectives, in each case justifying your answer. \\
  (i) Show $\{\neg, \to \}$ is a complete set of connectives. \\
  (ii) Show $\{\to, 0\}$ is a complete set of connectives (where $0$ is the constant false). \\
  (iii) Is $\{ \text{NAND}, \wedge\}$ a complete set of connectives? \\
  (iv) Is $\{\wedge, \vee\}$ a complete set of connectives?
\end{problem}

\subsection*{Answer}

In order to determine whether the sets are complete, I will be showing whether $\wedge, \vee$ and $\neg$ can be expressed using the connectives in the set, in which case any logical expression can be written in CNF or DNF, meaning that it's a complete set.

\begin{proof}[Proof for part (i)] $ $ \newline
  Can $\vee$ be expressed using $\{\neg, \to \}$? \\
  Yes: $\neg p \to q \equiv \neg(\neg p) \vee q \equiv p \vee q$ \\
  Can $\wedge$ be expressed using $\{\neg, to \}$? \\
  Yes: $\neg(p \to \neg q) \equiv \neg (\neg p \vee \neg q) \equiv p \wedge q$ \\
  Since $\neg$ is already in our set of logical connectives, we can then conclude that $\{\neg, \to \}$ is a complete set of logical connectives, as any logical expression can be expressed in CNF/DNF using the connectives within the set.
\end{proof}

\begin{proof}[Proof for part (ii)] $ $ \newline
  Can $\neg$ be expressed using $\{\to, 0\}$? \\
  Yes: $p \to 0 \equiv \neg p \vee 0 \equiv \neg p$ \\
  Can $\vee$ be expressed using $\{\to, 0\}$? \\
  Yes: $(p \to 0) \to q \equiv \neg (p \to 0) \vee q \equiv \neg (\neg p ) \vee q \equiv p \vee q$ \\
  Can $\wedge$ be expressed using $\{\to, 0\}$? \\
  Yes: $(p \to (q \to 0)) \to 0 \equiv (p \to (\neg q \vee 0)) \to 0 \equiv (p \to \neg q) \to 0 \equiv (\neg p \vee \neg q) \to 0 \equiv \neg(\neg p \vee \neg q) \vee 0 \equiv p \wedge q$ \\
  Therefore $\{\to, 0\}$ is a complete set of logical connectives
\end{proof}

\begin{proof}[Proof for part (iii)] $ $ \newline
  To denote NAND, I will use the symbol: $\barwedge$ \\
  Can $\neg$ be expressed using $\{\barwedge, \wedge\}$? \\
  Yes: $p \barwedge p \equiv \neg(p \wedge p) \equiv \neg p$ \\
  Can $\vee$ be expressed using $\{\barwedge, \wedge\}$? \\
  Yes: $(p \barwedge p) \barwedge (q \barwedge q) \equiv \neg p \barwedge \neg q \equiv \neg(\neg p \wedge \neg q) \equiv p \vee q$ \\
  $\wedge$ is already in the set of logical connectives, therefore the set of logical connectives $\{\text{NAND}, \wedge\}$ is complete.
\end{proof}

\begin{proof}[Proof for part (iv)] $ $ \newline
  The set of logical connectives $\{\wedge, \vee\}$ is not complete as there is no way to represent one of the propositional variables being equal to zero (i.e: negation) when writing an expression in CNF/DNF. That is to say there is no way to express a tautology or contradiction using these logical connectives due to their property of idempotence.
\end{proof}
\section*{Question 2}

\begin{problem}
  Convert $(((p \to q) \to r) \to (s \to t))$ to \\
  (i) Conjunctive Normal Form (CNF) \\
  (ii) Disjunctive Normal Form (DNF)
\end{problem}

\subsection*{Answer}

\ \newline
This question is easier to approach by writing the expression $\varphi = (((p \to q) \to r) \to (s \to t))$ in DNF first:
\begin{gather*}
  \varphi \equiv (((\neg p \vee q) \to r) \to (s \to t)) \\
  \equiv ((\neg(\neg p \vee q) \vee r) \to (s \to t)) \\
  \equiv (((p \wedge \neg q) \vee r) \to (s \to t)) \\
  \equiv (\neg((p \wedge \neg q) \vee r) \vee (s \to t)) \\
  \equiv ((\neg(p \wedge \neg q) \wedge \neg r) \vee (s \to t)) \\
  \equiv (((\neg p \vee q) \wedge \neg r) \vee (s \to t)) \\
  \equiv (((\neg p \wedge \neg r) \vee (q \wedge \neg r)) \vee (s \to t)) \\
  \equiv ((\neg p \wedge \neg r) \vee (q \wedge \neg r) \vee (\neg s \vee t)) \\
  \equiv (\neg p \wedge \neg r) \vee (q \wedge \neg r) \vee \neg s \vee t \\
  \therefore \varphi_{DNF} = (\neg p \wedge \neg r) \vee (q \wedge \neg r) \vee \neg s \vee t
\end{gather*}

Using this, we can repeatedly use the distributive property to convert this to conjunctive normal form:

\begin{gather*}
  \varphi_{DNF} = (\neg p \wedge \neg r) \vee (q \wedge \neg r) \vee \neg s \vee t \\
  \equiv ((\neg p \vee q) \wedge \neg r) \vee \neg s \vee t \\
  \equiv (((\neg p \vee q) \vee \neg s) \wedge (\neg r \vee \neg s)) \vee t \\
  \equiv (\neg p \vee q \vee \neg s \vee t) \wedge (\neg r \vee \neg s \vee t) \\
  \therefore \varphi_{CNF} = (\neg p \vee q \vee \neg s \vee t) \wedge (\neg r \vee \neg s \vee t)
\end{gather*}

(i) $\varphi_{CNF} = (\neg p \vee q \vee \neg s \vee t) \wedge (\neg r \vee \neg s \vee t)$

(ii) $\varphi_{DNF} =(\neg p \wedge \neg r) \vee (q \wedge \neg r) \vee \neg s \vee t$

\newpage

\section*{Question 3}

\begin{problem}
  What is the purpose of Tseitin's Algorithm? Apply Tseitin's Algorithm to turn the propositional formula $(((x_1 \wedge x_2 \wedge x_3) \to (y_1 \wedge y_2 \wedge y_3)) \vee z)$ to CNF.
\end{problem}

\subsection*{Answer}

The purpose of Tseitin's Algorithm is to take an arbitrary propositional formula $\varphi$, and transform it to a new propositional formula $\varphi'$ which is equisatisfiable with $\varphi$, and in conjunctive normal form.
\begin{gather*}
  \text{Let } \varphi = (((x_1 \wedge x_2 \wedge x_3) \to (y_1 \wedge y_2 \wedge y_3)) \vee z) \\
  \text{Introduce new variables for each subformula:} \\
  \alpha_1 \leftrightarrow x_1 \wedge x_2 \wedge x_3 \\
  \alpha_2 \leftrightarrow y_1 \wedge y_2 \wedge y_3 \\
  \alpha_3 \leftrightarrow z \\
  \alpha_4 \leftrightarrow \alpha_1 \to \alpha_2 \\
  \alpha_5 \leftrightarrow \alpha_4 \vee \alpha_3 \\
  \text{Write each expression as conjunctions} \\
  \text{From } \alpha_1: \\
  \alpha_1 \leftrightarrow (x_1 \wedge x_2 \wedge x_3) \equiv (\alpha_1 \to (x_1 \wedge x_2 \wedge x_3)) \wedge (\alpha_1 \leftarrow (x_1 \wedge x_2 \wedge x_3)) \\
  \equiv (\neg \alpha_1 \vee (x_1 \wedge x_2 \wedge x_3)) \wedge (\alpha_1 \vee \neg(x_1 \wedge x_2 \wedge x_3)) \\
  \equiv (\neg \alpha_1 \vee x_1) \wedge (\neg \alpha_1 \vee x_2) \wedge (\neg \alpha_1 \vee x_3) \wedge (\alpha_1 \vee \neg x_1 \vee \neg x_2 \vee \neg x_3) \\
  \text{Similarly for } \alpha_2: \\
  \alpha_2 \leftrightarrow (y_1 \wedge y_2 \wedge y_3) \equiv (\neg \alpha_2 \vee y_1) \wedge (\neg \alpha_2 \vee y_2) \wedge (\neg \alpha_2 \vee y_3) \wedge (\alpha_2 \vee \neg y_1 \vee \neg y_2 \vee \neg y_3) \\
  \text{For } \alpha_3: \\
  \alpha_3 \leftrightarrow z \equiv (\alpha_3 \to z) \wedge (\alpha_3 \leftarrow z) \\
  \equiv (\neg \alpha_3 \vee z) \wedge (\alpha_3 \vee \neg z) \\
  \text{For } \alpha_4: \\
  \alpha_4 \leftrightarrow \alpha_1 \to \alpha_2 \equiv (\alpha_4 \to (\alpha_1 \to \alpha_2)) \wedge (\alpha_4 \leftarrow  (\alpha_1 \to \alpha_2)) \\
  \equiv (\neg \alpha_4 \vee \neg \alpha_1 \vee \alpha_2) \wedge (\neg(\alpha_1 \to \alpha_2) \vee \alpha_4) \\
  \equiv (\neg \alpha_4 \vee \neg \alpha_1 \vee \alpha_2) \wedge (\neg(\neg \alpha_1 \vee \alpha_2) \vee \alpha_4) \\
  \equiv (\neg \alpha_4 \vee \neg \alpha_1 \vee \alpha_2) \wedge ((\alpha_1 \wedge \neg \alpha_2) \vee \alpha_4) \\
  \equiv (\neg \alpha_4 \vee \neg \alpha_1 \vee \alpha_2) \wedge (\alpha_1 \vee \alpha_4) \wedge (\neg \alpha_2 \vee \alpha_4) \\
  \text{For } \alpha_5: \\
  \alpha_5 \leftrightarrow \alpha_4 \vee \alpha_3 \equiv (\alpha_5 \to (\alpha_4 \vee \alpha_3)) \wedge (\alpha_5 \leftarrow (\alpha_4 \vee \alpha_3)) \\
  \equiv (\neg \alpha_5 \vee \alpha_4 \vee \alpha_3) \wedge (\neg(\alpha_4 \vee \alpha_3) \vee \alpha_5) \\
  \equiv (\neg \alpha_5 \vee \alpha_4 \vee \alpha_3) \wedge ((\neg \alpha_4 \wedge \neg \alpha_3) \vee \alpha_5) \\
  \equiv (\neg \alpha_5 \vee \alpha_4 \vee \alpha_3) \wedge (\neg \alpha_4 \vee \alpha_5) \wedge (\neg \alpha_3 \vee \alpha_5) \\
  \text{The conjunction of all these variables and the clause } \alpha_5 \\
  \text{ gives us the Tseitin Transformation of } \varphi' \\
  \text{To save space, I will write this as a clause set} \\
  \therefore \varphi' = \{\{\alpha_5\}, \{\neg \alpha_1, x_1\}, \{\neg \alpha_1, x_2\}, \{\neg \alpha_1, x_3\}, \{\alpha_1, \neg x_1, \neg x_2, \neg x_3\}, \{\neg \alpha_2, y_1\}, \{\neg \alpha_2, y_2,\} \\ \{\neg \alpha_2, y_3\}, \{\alpha_2, \neg y_1, \neg y_2, \neg y_3\}, \{\neg \alpha_3, z\}, \{\alpha_3, \neg z\}, \{\neg \alpha_4, \neg \alpha_1, \alpha_2\}, \{\alpha_1, \alpha_4\}, \{\neg \alpha_2, \alpha_4\}, \{\neg \alpha_5, \alpha_4, \alpha_3\} \\ \{\neg \alpha_4, \alpha_5\}, \{\neg \alpha_3. \alpha_5\}\}
\end{gather*}

\section*{Question 4}

\begin{problem}
  State with justification if each of the following sentences of predicate logic is logically valid. \\
  (i) $(\forall x \exists y \forall z \ E(x,y) \wedge E(y,z)) \to (\forall x \forall z \exists y \ E(x,y) \wedge E(y,z))$ \\
  (ii) $(\forall x \exists y \exists u \forall v \ E(x,y) \wedge E(u,v)) \to (\exists u \forall v \forall x \exists y \ E(x,y) \wedge E(u,v))$ \\
  (iii) $(\forall x \exists y \forall z \ R(x,y,z)) \to (\exists x \forall y \exists z \ R(x,y,z))$ \\
  (iv) $((\forall x \forall y \exists z (E(x,y) \wedge E(y,z))) \to (\forall x \forall y \forall z (E(x,y) \vee E(y,z))))$
\end{problem}

\section*{Question 5}

\begin{problem}
  Evaluate the given sentence on the respective relation $E$ over domain $\{0,1,2\}$ with relation $E := \{(0,1),(1,0),(1,2),(2,1),(2,0),(0,2)\}$\\ \\
  (i) $\forall x \forall y \forall z \exists w (E(x,w) \wedge E(y,w) \wedge E(z,w))$ \\
  (ii) $\exists x \forall y \forall z \exists w (E(x,w) \wedge E(y,w) \wedge E(z,w))$ \\
  (iii) $\forall y \exists x \forall z \exists w (E(x,w) \wedge E(y,w) \wedge E(z,w))$ \\
  (iv) $\exists x \exists y \exists z \forall w (E(x,w) \wedge E(y,w) \wedge E(z,w))$ \\
  (v) $\forall x_1 \exists x_2 \forall y_1 \exists y_2 \forall z_1 \exists z_2 \forall z \exists y \ E(x_1, x_2) \wedge E(x_2, w) \wedge E(y_1, y_2) \wedge E(y_2, w) \wedge E(z_1, z_2) \wedge E(z_2, w) \wedge E(z,w)$ \\
  (vi) $\forall x_1 \exists x_2 \forall y_1 \exists y_2 \forall z_1 \forall z \exists z_2 \exists y \ E(x_1, x_2) \wedge E(x_2, w) \wedge E(y_1, y_2) \wedge E(y_2, w) \wedge E(z_1, z_2) \wedge E(z_2, w) \wedge E(z,w)$
\end{problem}

\end{document}